\documentclass[conference]{IEEEtran}  % Formato tipo IEEE para conferencias

\usepackage[utf8]{inputenc}
\usepackage[T1]{fontenc}
\usepackage{cite}       % Para referencias IEEE
\usepackage{graphicx}   % Para incluir imágenes
\usepackage{amsmath}    % Para ecuaciones
\usepackage{hyperref}   % Para enlaces

\title{Título de tu Paper: Un Ejemplo de Lógica Combinatoria}

\author{
    \IEEEauthorblockN{Nombre del Autor}
    \IEEEauthorblockA{Instituto Tecnológico de Costa Rica\\
    Escuela de Ingeniería en Computadores\\
    CE1107 — Fundamentos de Arquitectura de Computadores\\
    Correo Electrónico: ejemplo@tec.ac.cr}
}

\begin{document}

\maketitle

\begin{abstract}
Este abstract es un ejemplo. Un buen abstract permite a los lectores obtener la esencia de su artículo rápidamente, preparar a los lectores para seguir la información detallada y ayudar a recordar los puntos clave. Debe estar escrito en inglés y tener entre 150 y 250 palabras. Aquí se resumirá brevemente el objetivo del trabajo, la metodología utilizada y los principales resultados obtenidos.
\end{abstract}

\begin{IEEEkeywords}
Lógica combinatoria, algoritmo, resultados, conclusión, recomendación, IEEE.
\end{IEEEkeywords}

\section{Introducción}
La introducción debe presentar el contexto del problema o tema que se va a resolver, explicando su relevancia y antecedentes. Al final de la introducción se describe la organización del documento, por ejemplo: en la sección II se presenta el algoritmo desarrollado, en la sección III los resultados obtenidos, en la sección IV conclusiones y en la sección V recomendaciones.

\section{Algoritmo Desarrollado}
Aquí se describe detalladamente el algoritmo que se implementó para resolver el problema. Se pueden incluir diagramas de flujo, pseudocódigo o ecuaciones relevantes.  
\begin{equation}
Y = A \cdot B + \overline{C}
\end{equation}

\section{Resultados}
En esta sección se presentan los resultados obtenidos tras la implementación del algoritmo, con tablas, gráficas o comparaciones según sea necesario.  
\begin{figure}[h]
    \centering
    \includegraphics[width=0.45\textwidth]{images/74LS157.png}
    \caption{Ejemplo de resultado gráfico.}
    \label{fig:resultados}
\end{figure}

\section{Conclusiones}
Las conclusiones se redactan en prosa, indicando los hallazgos más importantes, la efectividad del algoritmo y su aplicabilidad.

\section{Recomendaciones}
Se escriben recomendaciones sobre posibles mejoras, aplicaciones futuras o precauciones a tener en cuenta.

\section*{Bibliografía}
\bibliographystyle{IEEEtran}
\bibliography{referencias} % archivo referencias.bib con formato BibTeX

\end{document}

