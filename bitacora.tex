% Bitacora_Proyecto.tex
% Plantilla LaTeX para la bitácora de un proyecto (Conocimiento de Ingeniería) — 10%
% Instrucciones: compilar con pdflatex o latexmk. Mantener el archivo en un repositorio git y hacer commits frecuentes.
\documentclass[12pt,a4paper]{article}
\usepackage[utf8]{inputenc}
\usepackage[T1]{fontenc}
\usepackage[spanish]{babel}
\usepackage{geometry}
\usepackage{hyperref}
\usepackage{fancyhdr}
\usepackage{booktabs}
\usepackage{longtable}
\usepackage{array}
\usepackage{amsmath,amssymb}
\usepackage{listings}
\usepackage{graphicx}
\usepackage{datetime}
\usepackage{xcolor}
\geometry{margin=2.5cm}

% Cabecera y pie
\pagestyle{fancy}
\fancyhf{}
\lhead{Bitácora — Conocimiento de Ingeniería}
\rhead{\textbf{Proyecto:} Nombre del Proyecto}
\cfoot{Página \thepage}

% Comandos útiles
\newcommand{\entry}[4]{%
  \subsection*{#1}
  \textbf{Fecha:} #2 \\
  \textbf{Participantes:} #3 \\
  \textbf{Objetivos del día:} \\
  #4 \\
  \vspace{6pt}
}

\begin{document}

\begin{titlepage}
  \centering
  {\scshape\LARGE Universidad / Instituto \par}
  \vspace{1cm}
  {\huge\bfseries Bitácora de Proyecto: Nombre del Proyecto \par}
  \vspace{1.5cm}
  {\Large \textit{Atributo evaluado:} Conocimiento de Ingeniería (10\%) \par}
  \vfill
  {\large Autor: Nombre del Estudiante \par}
  {\large Curso: Nombre del Curso \par}
  {\large Profesor: Nombre del Profesor \par}
  \vspace{1cm}
  {\large Fecha de inicio: \today \par}
\end{titlepage}

\tableofcontents
\newpage

\section{Instrucciones de uso}
Esta bitácora debe actualizarse periódicamente. Cada entrada debe corresponder a una fecha (o rango de fechas) e incluir:
\begin{itemize}
  \item Actividades realizadas y procedimientos implementados (paso a paso).
  \item Tablas de datos experimentales o de diseño.
  \item Ecuaciones booleanas, mapas de Karnaugh y simplificaciones (si aplica).
  \item Herramientas de automatización y software utilizados (versiones y comandos relevantes).
  \item Enlaces a commits relevantes en el repositorio git (hash corto y mensaje).
\end{itemize}

\section{Resumen ejecutivo}
Breve resumen del progreso del proyecto, hitos y estado actual.

\section{Registro diario / por sesión}
% --- Ejemplo de entrada ---
\subsection*{Plantilla de entrada}
\noindent\textbf{Fecha:} \underline{YYYY-MM-DD}\\
\textbf{Participantes:} \underline{Nombre(s)}\\
\textbf{Objetivos del día:} \\ 
\textbf{Procedimiento y pasos realizados:} \\ 
\begin{enumerate}
  \item Paso 1: descripción detallada. Incluir comandos, parámetros y valores medidos.
  \item Paso 2: ...
\end{enumerate}
\textbf{Resultados (tablas, gráficas):} \\ 
\textbf{Ecuaciones y simplificaciones (si aplica):} 
\begin{verbatim}
  -- Aquí incluir ecuaciones booleanas o algebraicas
\end{verbatim}
\textbf{Problemas encontrados y cómo se resolvieron:} \\ 
\textbf{Tareas para la próxima sesión:} \\ 
\vspace{0.5cm}

% --- Entrada de ejemplo rellena ---
\subsection*{Ejemplo: Semana 1 - Diseño inicial}
\textbf{Fecha:} 2025-09-01\\
\textbf{Participantes:} Jafet Díaz Morales\\
\textbf{Objetivos del día:} Definir requerimientos, seleccionar microcontrolador y elaborar diagrama inicial.

\textbf{Procedimiento y pasos realizados:}
\begin{enumerate}
  \item Revisión de requisitos del curso y alcance del proyecto.
  \item Selección de Raspberry Pi Pico W por disponibilidad y conectividad Wi-Fi.
  \item Diseño preliminar del circuito en esquemático (ver Figura~\ref{fig:esquematico}).
\end{enumerate}

\begin{figure}[h]
  \centering
  % \includegraphics[width=0.7\textwidth]{esquematico.png} % descomentar y agregar archivo
  \caption{Esquemático preliminar (añadir archivo si aplica).}
  \label{fig:esquematico}
\end{figure}

\textbf{Resultados (tabla de componentes):}
\begin{center}
\begin{tabular}{@{}llr@{}}
\toprule
Referencia & Componente & Cantidad \\
\midrule
R1 & Resistor 10k & 2 \\
C1 & Condensador 10uF & 1 \\
U1 & Raspberry Pi Pico W & 1 \\
\bottomrule
\end{tabular}
\end{center}

\textbf{Ecuaciones booleanas y simplificación:}
\begin{align*}
  F(A,B,C) &= A B + \overline{A} C \\ 
  \text{Mapa de Karnaugh: simplificar a } F &= B A + \overline{A}C \quad(\text{ejemplo}).
\end{align*}

\textbf{Problemas encontrados y soluciones:} Lecturas del sensor ruidosas; se añadió un filtro RC (R=10k, C=100nF) y se promediaron lecturas.

\textbf{Commits relevantes:}
\begin{itemize}
  \item \texttt{e3a1b2c} - "Inicial: plantilla del proyecto y esquemático".
  \item \texttt{f4d5e6a} - "Agregar tabla de componentes y ejemplo de bitácora".
\end{itemize}

\newpage
\section{Tablas de datos}
Incluya aquí tablas largas usando \texttt{longtable} cuando sea necesario.

\begin{longtable}{@{}lll@{}}
\caption{Registro de mediciones (ejemplo)}\\
\toprule
Fecha & Parámetro & Valor \\
\midrule
\endfirsthead
\toprule
Fecha & Parámetro & Valor \\
\midrule
\endhead
2025-09-01 & Voltaje VCC & 3.3 V \\
2025-09-02 & Corriente pico & 120 mA \\
\bottomrule
\end{longtable}

\section{Ecuaciones y diseño lógico}
Usar secciones separadas para presentar la lógica booleana, tablas de verdad, mapas de Karnaugh y la simplificación final.

\subsection*{Tabla de verdad (ejemplo)}
\begin{center}
\begin{tabular}{cccc}
\toprule
A & B & C & F \\
\midrule
0 & 0 & 0 & 0 \\
0 & 0 & 1 & 1 \\
0 & 1 & 0 & 0 \\
0 & 1 & 1 & 1 \\
1 & 0 & 0 & 0 \\
1 & 0 & 1 & 0 \\
1 & 1 & 0 & 1 \\
1 & 1 & 1 & 1 \\
\bottomrule
\end{tabular}
\end{center}

\section{Herramientas y automatización}
Describa aquí las herramientas (hardware y software) y los comandos de automatización empleados. Incluya versiones y fragmentos de comandos.

\subsection*{Ejemplo de herramientas}
\begin{itemize}
  \item Sistema operativo: Ubuntu 22.04 (o macOS 14)
  \item IDE: Visual Studio Code v1.xx
  \item Lenguaje / Framework: MicroPython / C++ (indicar versiones)
  \item Herramientas de automatización: Make, CMake, scripts bash, GitHub Actions (añadir workflows si aplica).
\end{itemize}

\subsection*{Comandos útiles (ejemplo)}
\begin{lstlisting}
# Inicializar repositorio
git init
git add Bitacora_Proyecto.tex
git commit -m "Bitácora: plantilla inicial"
# Subir a remoto (ejemplo GitHub)
git remote add origin git@github.com:usuario/repositorio.git
git push -u origin main

# Compilar PDF
pdflatex Bitacora_Proyecto.tex
# O con latexmk
latexmk -pdf Bitacora_Proyecto.tex
\end{lstlisting}

\section{Historial de commits (muestra)}
Aquí debe pegarse el historial de commits relevante (hash, fecha y mensaje). Ejemplo:
\begin{verbatim}
commit e3a1b2c 2025-09-01 "Inicial: plantilla del proyecto y esquemático"
commit f4d5e6a 2025-09-02 "Agregar tabla de componentes y ejemplo de bitácora"
\end{verbatim}

\section{Anexos}
Incluya diagramas, fotografías del montaje, esquemáticos y cualquier otro material complementario. Añadir referencias a archivos en el repositorio (ruta relativa).

\bigskip
\noindent\textbf{Notas finales:}
\begin{itemize}
  \item Mantenga commits regulares: al menos 1 commit por sesión de trabajo (se penalizarán pocas confirmaciones).
  \item Use mensajes de commit claros y vinculantes con la entrada de la bitácora (ej.: "Bitácora: pruebas sensor DHT11 — 2025-09-03").
  \item Entregue el PDF generado y el enlace al repositorio con todo el historial.
\end{itemize}

\end{document}
